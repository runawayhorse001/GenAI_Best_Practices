%% Generated by Sphinx.
\def\sphinxdocclass{report}
\documentclass[letterpaper,11pt,english]{sphinxmanual}
\ifdefined\pdfpxdimen
   \let\sphinxpxdimen\pdfpxdimen\else\newdimen\sphinxpxdimen
\fi \sphinxpxdimen=.75bp\relax

\PassOptionsToPackage{warn}{textcomp}
\usepackage[utf8]{inputenc}
\ifdefined\DeclareUnicodeCharacter
% support both utf8 and utf8x syntaxes
  \ifdefined\DeclareUnicodeCharacterAsOptional
    \def\sphinxDUC#1{\DeclareUnicodeCharacter{"#1}}
  \else
    \let\sphinxDUC\DeclareUnicodeCharacter
  \fi
  \sphinxDUC{00A0}{\nobreakspace}
  \sphinxDUC{2500}{\sphinxunichar{2500}}
  \sphinxDUC{2502}{\sphinxunichar{2502}}
  \sphinxDUC{2514}{\sphinxunichar{2514}}
  \sphinxDUC{251C}{\sphinxunichar{251C}}
  \sphinxDUC{2572}{\textbackslash}
\fi
\usepackage{cmap}
\usepackage[T1]{fontenc}
\usepackage{amsmath,amssymb,amstext}
\usepackage{babel}



\usepackage{times}
\expandafter\ifx\csname T@LGR\endcsname\relax
\else
% LGR was declared as font encoding
  \substitutefont{LGR}{\rmdefault}{cmr}
  \substitutefont{LGR}{\sfdefault}{cmss}
  \substitutefont{LGR}{\ttdefault}{cmtt}
\fi
\expandafter\ifx\csname T@X2\endcsname\relax
  \expandafter\ifx\csname T@T2A\endcsname\relax
  \else
  % T2A was declared as font encoding
    \substitutefont{T2A}{\rmdefault}{cmr}
    \substitutefont{T2A}{\sfdefault}{cmss}
    \substitutefont{T2A}{\ttdefault}{cmtt}
  \fi
\else
% X2 was declared as font encoding
  \substitutefont{X2}{\rmdefault}{cmr}
  \substitutefont{X2}{\sfdefault}{cmss}
  \substitutefont{X2}{\ttdefault}{cmtt}
\fi


\usepackage[Bjarne]{fncychap}
\usepackage{sphinx}

\fvset{fontsize=\small}
\usepackage{geometry}


% Include hyperref last.
\usepackage{hyperref}
% Fix anchor placement for figures with captions.
\usepackage{hypcap}% it must be loaded after hyperref.
% Set up styles of URL: it should be placed after hyperref.
\urlstyle{same}


\usepackage{sphinxmessages}
\setcounter{tocdepth}{2}



\title{GenAI: Best Practices}
\date{December 05, 2024}
\release{1.0}
\author{Wenqiang Feng and Di Zhen}
\newcommand{\sphinxlogo}{\sphinxincludegraphics{logo.png}\par}
\renewcommand{\releasename}{Release}
\makeindex
\begin{document}

\pagestyle{empty}
\sphinxmaketitle
\pagestyle{plain}
\sphinxtableofcontents
\pagestyle{normal}
\phantomsection\label{\detokenize{index::doc}}\phantomsection\label{\detokenize{index:index}}\begin{quote}

\begin{figure}[htbp]
\centering

\noindent\sphinxincludegraphics{{logo}.png}
\end{figure}
\end{quote}

Welcome to our \sphinxstylestrong{GenAI: Best Practices}!!! The PDF version
can be downloaded from \sphinxhref{../latex/GenAI.pdf}{HERE}.




\chapter{Preface}
\label{\detokenize{preface:preface}}\label{\detokenize{preface:id1}}\label{\detokenize{preface::doc}}
\begin{sphinxadmonition}{note}{Chinese proverb}

Good tools are prerequisite to the successful execution of a job. \textendash{} old Chinese proverb
\end{sphinxadmonition}


\section{About}
\label{\detokenize{preface:about}}

\subsection{About this API}
\label{\detokenize{preface:about-this-api}}
This document is the \sphinxcode{\sphinxupquote{API}} book for our AutoFeatures: PySpark Auto Feature Selector \sphinxcite{reference:autofeatures} API.
The PDF version can be downloaded from \sphinxhref{GenAI.pdf}{HERE}. \sphinxstylestrong{You may download and distribute it. Please beaware,
however, that the note contains typos as well as inaccurate or incorrect description.}

The \sphinxcode{\sphinxupquote{API}} assumes that the reader has a preliminary knowledge of \sphinxcode{\sphinxupquote{python}} programing and \sphinxcode{\sphinxupquote{Linux}}. And this
document is generated automatically by using \sphinxhref{http://sphinx.pocoo.org}{sphinx}.


\subsection{About the author}
\label{\detokenize{preface:about-the-author}}\begin{itemize}
\item {} 
\sphinxstylestrong{Wenqiang Feng}
\begin{itemize}
\item {} 
Sr. Data Scientist and PhD in Mathematics

\item {} 
University of Tennessee at Knoxville

\item {} 
Webpage: \sphinxurl{http://web.utk.edu/~wfeng1/}

\item {} 
Email: \sphinxhref{mailto:von198@gmail.com}{von198@gmail.com}

\end{itemize}

\item {} 
\sphinxstylestrong{Biography}

Wenqiang Feng is Data Scientist within DST’s Applied Analytics Group. Dr. Feng’s responsibilities include providing
DST clients with access to cutting\sphinxhyphen{}edge skills and technologies, including Big Data analytic solutions, advanced
analytic and data enhancement techniques and modeling.

Dr. Feng has deep analytic expertise in data mining, analytic systems, machine learning algorithms, business
intelligence, and applying Big Data tools to strategically solve industry problems in a cross\sphinxhyphen{}functional business.
Before joining DST, Dr. Feng was an IMA Data Science Fellow at The Institute for Mathematics and its
Applications (IMA) at the University of Minnesota. While there, he helped startup companies make marketing
decisions based on deep predictive analytics.

Dr. Feng graduated from University of Tennessee, Knoxville, with Ph.D. in Computational Mathematics and Master’s
degree in Statistics. He also holds Master’s degree in Computational Mathematics from Missouri University of
Science and Technology (MST) and Master’s degree in Applied Mathematics from the University of Science and
Technology of China (USTC).

\item {} 
\sphinxstylestrong{Declaration}

The work of Wenqiang Feng was supported by the IMA, while working at IMA. However, any opinion, finding,
and conclusions or recommendations expressed in this material are those of the author and do not necessarily
reflect the views of the IMA, UTK and DST.

\end{itemize}


\section{Feedback and suggestions}
\label{\detokenize{preface:feedback-and-suggestions}}
Your comments and suggestions are highly appreciated. I am more than happy to receive
corrections, suggestions or feedback through email (Wenqiang Feng: \sphinxhref{mailto:von198@gmail.com}{von198@gmail.com}) for improvements.


\chapter{Main Reference}
\label{\detokenize{reference:main-reference}}\label{\detokenize{reference:reference}}\label{\detokenize{reference::doc}}
\begin{sphinxthebibliography}{AutoFeat}
\bibitem[AutoFeatures]{reference:autofeatures}
Wenqiang Feng and Ming Chen.
\sphinxhref{https://runawayhorse001.github.io/AutoFeatures/}{Python Data Audit Library API}, 2019.
\end{sphinxthebibliography}



\renewcommand{\indexname}{Index}
\printindex
\end{document}